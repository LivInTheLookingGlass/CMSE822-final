\documentclass{article}
\usepackage[utf8]{inputenc}

\title{CMSE 822 Final Project }
\author{Carolyn Wendeln}
\date{\today}

\begin{document}

\maketitle


Modern astrophysical research has become increasingly dominated by large scale, multi-dimensional simulations.
Phenomena such as supernovae core collapse, coronal mass ejections, or neutron star mergers all rely on multi-scale simulations.
These simulations can span orders of magnitude in scale separation, yet these vast scales are required to resolve the key underlying physics involved.
Although these large scale simulations have become a necessity, they often require an ever-increasing amount of computational resources.

One key feature that most of these astrophysical phenomena have in common is that they are often coupled to hyperbolic partial differential equations (PDEs).
Equations such as Euler, Cauchy, or Navier–Stokes are often used for simulating astrophysical plasmas.
These simulations often call for techniques such as adaptive mesh refinement (AMR) and high order finite difference or finite volume methods.

For this research project, I will be investigating second and fourth order discretization of hyperbolic PDEs.
In particular, I will look at the two dimensional form of the advection equation, viscous Burgers' equation, and inviscid Burgers' equation.
This work will focus on these equations over a periodic domain on a uniform mesh grid using finite difference methods.

The goal of this research project is to understand the underlying issues surrounding hyperbolic solvers for fluid equations.
I will investigate various optimization techniques for a single node.
My work will begin with parallelization using OpenMP 4.5 or 5.0. to target both CPUs and GPUs.
After parallelisation, I will employ other techniques such as cache blocking to see the effects of strong and weak scaling.
Once I believe I have achieved optimal performance using OpenMP, I will try to implement similar techniques using Kokkos.

Kokkos is an open source c++ performance portability programming model that has recently been coupled with Athena++ to create K-Athena \footnote[1]{https://ieeexplore.ieee.org/document/9143480}.
Athena++ is a magnetohydrodynamics CUP code that is based on MPI+OpenMP.
K-Athena utilizes MPI for inter-node parallelism, hence my project here will focus solely on optimization for a single node.
Likewise Kokkos is compatible with optimization techniques which target both CPUs and GPUs.
K-Athena will become a key component to my own thesis, so this research project serves as a fantastic stepping stone to my future research studying astrophysical phenomena using AMR and finite volume methods.






 

\end{document}